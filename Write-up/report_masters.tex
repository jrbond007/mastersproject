\documentclass[letterpaper,11pt]{article}
\usepackage[margin=1.5in]{geometry}
\usepackage[english]{babel}
%\usepackage{blindtext}
\usepackage{multicol}
\usepackage{fullpage}
\usepackage{mathtools}
\usepackage{xcolor}
%\usepackage[left=3cm,top=3cm,right=3cm,nohead,nofoot]{geometry}
\usepackage{amsmath, amssymb, graphicx, color, array, appendix}
\usepackage[T1]{fontenc}
\usepackage[utf8]{inputenc}
\usepackage{sistyle}
\usepackage{caption}
\usepackage{soul}
%\usepackage{hyperref}
\usepackage{listings}
\usepackage{verbatim}
\usepackage{calc}
\usepackage{graphics}
\usepackage{wrapfig}
\usepackage{amsthm}
\usepackage{amsmath}
\lstset{language=C}

\linespread{1.3}
%\setlength{\hoffset}{0.5in}
\newcommand{\ud}{\,\mathrm{d}}

%Title section
\begin{document}
%\maketitle
%\tableofcontents
\begin{abstract}
%Write an abstract
A model of inflation with both a longitudinal and a transverse field is explored in which the longitudinal field drives inflation while the transverse field experiences a temporary instability. The evolution of the system during inflation is calculated by making use of a lattice simulation. The change in $\zeta$ as sourced by the gradient terms in the transverse field generated during its instability is calculated.
\end{abstract}
\section{Introduction}
\section{Theory}
%Makes sense to start with the FRW stuff as it is used below. Can also motivate the form of the action as being the sum of the Einstein-Hilbert action, and scalar field action with a coupling term.
%note that this is the flat FRW metric
The simplest that can be considered for the universe is one in which the universe is spatially homogeneous and isotropic. It is well known that such a universe can be described by a Friedmann-Robertson-Walker (FRW) metric with line element
\begin{equation}
\ud^2s=-\ud^2t+a^2(t)(\ud x^2+\ud y^2+\ud z^2). \label{line element}
\end{equation}
In (\ref{line element}) $x$, $y$, $z$ are known as comoving coordinates while $a(t)$ is the scale factor. The physical significance of these quantities is that a test particle initially at rest at one position (as measured in comoving coordinates) will remain at rest at that position (again, as measured in comoving coordinates) while the physical distance between any two such stationary test particles is proportional to the scale factor and will evolve in time.

The time evolution of a universe described by a FRW metric is determined by substituting the metric associated with (\ref{line element}) into the Einstein equation, the result is the Friedmann equation
\begin{equation}
H^2=\frac{8\pi G}{3}\rho. \label{fried eqn}
\end{equation}
We have introduced $H\equiv \dot{a}/a$ which is known as the Hubble parameter and measures the rate of expansion of the universe. An important property of the Hubble parameter is its inverse, $H^-1$, provides the relevant length scale for regions which are causally connected.%Reference this statement

Another relation which proves useful and takes a simple for in a FRW universe is the fluid equation which is arrived at by equating the rate of change of energy in a unit comoving volume to the rate at which work is done on that volume
\begin{align}
&\frac{\ud}{\ud t}(\rho a^3)=-p\frac{\ud}{\ud t}(a^3),\\
&\dot{\rho}+3H(\rho +p)=0. \label{fluid eqn}
\end{align}

%Talk about cosmic horizon

%Introduce inflation


We begin by considering the action of a set of scalar fields coupled to gravity,
\begin{equation}
S=\int \Big\{ \frac{R}{16\pi} - \frac{1}{2}g^{\mu \nu} \phi_{,\mu}^{\mathrm{A}} \phi_{,\nu}^{\mathrm{A}}-V(\phi^{\mathrm{A}}) \Big\} \sqrt{-g}\ud^4x
\end{equation}
where $R$ is the Ricci scalar, $g^{\mu \nu}$ is the metric tensor, $g$ is the determinant of the metric tensor, uppercase Roman indices run over the set of scalar fields, sums over repeated indices being implied. %Talk about dropping term for minimal coupling

In this report we will be mainly with concerned with the case of two scalar fields minimally coupled to gravity. It is assumed that these fields are sufficiently close to being uniform and isotropic that the metric tensor is well approximated a Friedmann-Robertson-Walker type metric with a line element of the form $\ud s^2=-\ud t^2+a^2(t)(\ud x^2+\ud y^2+\ud z^2)$. With these assumptions the action can be rewritten as
\begin{equation}
S=\int \Big\{ \frac{R}{16 \pi}+\frac{1}{2a^2}(a^2 \dot{\phi}^2+a^2\dot{\chi}^2-(\nabla{\phi})^2-(\nabla{\chi})^2)-V(\phi,\chi) \Big\}a^3\ud t\ud^3x.
\end{equation}
Applying the Euler-Lagrange condition to extremize the above action results in the following equations of motion for the scalar fields:
\begin{align}
0&=\ddot{\phi}+3H\dot{\phi}+(-\frac{\Delta \phi}{a^2}+V_{,\phi}),\\
0&=\ddot{\chi}+3H\dot{\chi}+(-\frac{\Delta \chi}{a^2}+V_{,\chi}).
\end{align}
Two notable differences between these equations of motion and their Minkowski space counterparts is the addition of a drag term proportional to the Hubble parameter, and the scaling of the Laplacian by a factor of $a^{-2}$. The first of these differences is dubbed the Hubble drag term and results from the time dependence of $a$. The second of these differences can be intuitively understood as the need to differentiate with respect to physical distances instead of coordinate distances.

%bit about the R/16pi term giving Einstein's equation
It is further useful to put the fields $\phi$, $\chi$ and potential $V(\phi ,\chi )$ in units of the reduced Planck mass defined $M_{Pl} \equiv (8 \pi G)^-{1/2}$. 

%State the pressure and density of a scalar field and introduce the reduced Planck mass
%Relate to the Friedmann equation and introduce inflation
%Reference inflation and say a bit about the physical significance of having an inflationary phase

Inflation was originally proposed as a possible solution to the horizon and flatness problems %Describe problems and cite Guth

We can take the defining feature of inflation to be an accelerating expansion, 
\begin{equation}
\ddot{a}>0.
\end{equation}
Recalling the definition of $H$ it is easily verified that this condition of accelerating expansion can be expressed equivalently as
\begin{equation}
\frac{\ud}{\ud t}(\frac{H^{-1}}{a})<0.
\end{equation}
The physical significance of inflation can be extracted from this second definition by recognizing $H^{-1}$ as the relevant distance scale of the cosmic horizon while $a$ is the relevant distance scale between a pair of comoving points. As this ratio of scales decreases the cosmic horizon effectively shrinks with respect to comoving coordinates so that two comoving points initially causally connected may cease to be causally connected at some later time during inflation. Once inflation has terminated %write about re entering the horizon

%bit about inflation explaining structure

So far we have described inflation in terms of the scale factor and Hubble parameter, but have not described a scenario that would lead to inflation actually taking place and would like to relate such a scenario back to scalar fields. To relate inflation back to scalar fields we first derive a condition for inflation to proceed. By differentiating both sides of the Friedmann equation \ref{fried eqn} and substituting using the continuity equation \ref{fluid eqn} and Friedman equation \ref{fried eqn} it is possible to give conditions on when inflation will occur.%reference Liddle (check)
\begin{align}
\frac{\ud}{\ud t}(H^2)&=\frac{\ud}{\ud t}(\frac{1}{3}\rho)\\
\frac{\ddot{a}}{a}&=\frac{\dot{\rho}}{6H}+H^2\\
\frac{\ddot{a}}{a}&=-\frac{1}{2}(\frac{1}{3}\rho+p)
\end{align}
%I have switched to using the reduced Planck mass, make sure I include an explination for this
The above calculation immediately gives the necessary and sufficient condition for inflation to occur in a FRW universe that $p+\rho/3<0$. For the case of inflation driven by single, spatially homogeneous, scalar field this condition can be rewritten using (\ref{density}) and (\ref{pressure})
\begin{equation}
\dot{\phi}^2-V(\phi)<0. \label{inflation condition 3a}
\end{equation}

%need to talk about zeta

\section{Computation Techniques}
Calculations solving the equations of motion the system are performed using a lattice simulation code %citation to code
The lattice simulation solves for the evolution of a number of scalar fields from initial conditions as well as solving for the scale factor and Hubble factor assuming a FRW metric. For each site on the lattice values of the matter fields are stored, whereas the scale factor and Hubble parameter are assumed to be uniform over the simulation box. Derivative terms are calculated using a pseudo spectral method whereby a fast Fourier transform (FFT) is taken of the matter field values over the simulation box, then derivatives are then calculated in Fourier and finally the FFT is inverted to return the derivatives to real space.

The equations of motion are cast into a Hamiltonian framework and the resulting system of equations discretized so that each point of the lattice is associated with its own conjugate variables. We have from Hamiltonian mechanic

The system of equations so discretized , Hamilton's equations are put into a matrix equation format and finding the solutions to the equations of motion is equivalent to integrating this matrix equation. The 

\section{}
In standard inflationary scenarios inflation is driven by a single scalar field which is initially homogeneous apart from small fluctuations and far from the minimum of its potential. %quantify this with V'/V
For definiteness we will reserve $\phi$ to refer to the inflaton field, that is the field driving inflation.
As this scalar field evolves following (\ref{eom phi}) it will initially proceed with $\ddot{\phi} \approx 0$, this situation is analogous to reaching terminal velocity with the Hubble parameter playing the role of drag. During this time $\dot{\phi}^2$ will be limited in magnitude by the $V_{,\phi}$ term in (\ref{eom phi}) and for large enough values of $V(\phi)$ the dominant contribution to the energy of the system will be from the potential, ensuring (\ref{inflation condition 3a}) is satisfied. It should be noted this description of how inflation can be driven by a scalar field contain several assumptions that may fail to hold towards the end of inflation and does not form a definition of inflation. A calculation of the evolution of such an inflaton field during inflation and shortly after the end of inflation is shown in \ref{inflaton figure}.

After the end of inflation as the inflaton approaches the minimum of its potential and (\ref{eom phi}) becomes a damped oscillator equation (albeit with a drag term which has its own equation of motion given by \ref{fried eqn}). So, after inflation the inflaton will oscillate around the minimum of its potential. A calculation of the evolution of such an inflaton field during inflation and shortly after the end of inflation is shown in \ref{inflaton figure}.

For definiteness we will take $\phi>0$ at initial conditions, then noting that $\phi$ decreases monotonically during inflation (see \ref{inflation figure}) $\phi$ can be thought of as a decreasing time-like variable during inflation.

In what follows, we will consider an extension to the above inflationary scenario by including a second scalar field, $chi$. In the case we consider inflation is driven by the $\phi$ field (termed longitudinal), while the  $\chi$ field (termed transverse) is composed of fluctuations around zero. The potential considered is,
\begin{equation}
V(\phi, \chi) = \frac{\lambda_{\phi}}{4}\phi^4 + \frac{\lambda_{\chi}}{4}\chi^4 + \Delta V, \label{potential}
\end{equation}
where $\Delta V$ a temporary instability in the transverse direction. The form chosen for $\Delta V$ is given by
\begin{equation}
\Delta V = -\frac{A^2}{b\sqrt(e)}(\phi - \phi_p)exp[-\frac{(\phi-\phi_p)^2}{2b^2}]\chi^2.
\end{equation}
The parameters $A^2$, $b$, and $\phi_p$ respectively control the magnitude, width, and location of $\Delta V$. It can be noted that this form of $\Delta V$ is an effective mass term for the $\chi$ field. When $|\phi-\phi_p| \gg b$ $\Delta V \approx 0$ and has no effect. As $\phi$ approaches $\phi_p$ from above the $\chi$ field acquires a negative effective mass which reaches a minimum at $\phi = \phi_p + b$ (the potential remaining bounded from below by the $\chi^4$ term in \ref{potential}). Likewise, as $\phi$ approaches $\phi_p$ from below the $\chi$ field acquires a positive effective mass which reaches a maximum at $\phi = \phi_p - b$. A plot showing the effective mass bestowed on $\chi$ is shown in \ref{m2eff plot}. 

There is associated with the instability in $\chi$ caused by $\Delta V$ an instability in $\phi$ and if we wish to use $\phi$ as a time-like variable during inflation it should be verified that it remains monotonic for cases with non-zero $\Delta V$. Although it is possible to violate the monotonicity with a large enough $\Delta V$, in practice the monotonicity condition is easy to satisfy due to our choice of $\phi$ as longitudinal and $\chi$ as transverse. This choice means that during inflation $|\phi| \gg |\chi|$ will be satisfied and the sign of $V_{\phi}$ will not change for small enough $\Delta V$.

The effect of a non-zero $\Delta V$ in the potential is that $\chi$ will experience a temporary tachyonic instability as $\phi$ passes through the point $\phi=\phi_p$ while driving inflation. During this instability the $\chi$ field will experience exponential growth away from zero, with points where the fluctuations of $\chi$ are positive will exponentially increase, while points where the fluctuations of $\chi$ are negative will exponentially become more negative. The net effect of this instability is to cause a separation between trajectories of initially similar fluctuations in $\chi$. As $\phi$ decrease below $\phi_p$ the instability is terminated and the separation of adjacent trajectories ceases. This separation of trajectories is illustrated in \ref{chi_v-chi_nov plot} which shows the difference between several sampled trajectories of $\chi$ with and without $\Delta V$.











\end{document}